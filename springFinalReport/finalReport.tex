\documentclass[onecolumn, draftclsnofoot,10pt, compsoc]{IEEEtran}
\usepackage{graphicx}
\usepackage{url}
\usepackage{setspace}
\usepackage{array}
\usepackage{tabu}
\usepackage{listings}
\usepackage{geometry}
\geometry{textheight=9.5in, textwidth=7in}
\graphicspath{ {images/} }
% 1. Fill in these details
\def \CapstoneTeamName{		Deep Learning on Embedded Platform}
\def \CapstoneTeamNumber{		14}
\def \GroupMemberOne{			Christopher Johnson}
\def \GroupMemberTwo{			Gabe Morey}
\def \GroupMemberThree{			Luay Alshawi}
\def \CapstoneProjectName{		AI Gaming}
\def \CapstoneSponsorCompany{	NVIDIA}
\def \CapstoneSponsorPerson{		Mark Ebersole}

\def \DocType{	%Problem Statement
				%Requirements Document
				%Technology Review
				%Design Document
				Progress Report
				}

\newcommand{\NameSigPair}[1]{\par
\makebox[2.75in][r]{#1} \hfil 	\makebox[3.25in]{\makebox[2.25in]{\hrulefill} \hfill		\makebox[.75in]{\hrulefill}}
\par\vspace{-12pt} \textit{\tiny\noindent
\makebox[2.75in]{} \hfil		\makebox[3.25in]{\makebox[2.25in][r]{Signature} \hfill	\makebox[.75in][r]{Date}}}}

\begin{document}
\begin{titlepage}
    \pagenumbering{gobble}
    \begin{singlespace}
    	% \includegraphics[height=4cm]{coe_v_spot1}
        \hfill
        % 4. If you have a logo, use this includegraphics command to put it on the coversheet.
        %\includegraphics[height=4cm]{CompanyLogo}
        \par\vspace{.2in}
        \centering
        \scshape{
            \huge CS Capstone \DocType \par
            {\large\today}\par
            \vspace{.5in}
            \textbf{\Huge\CapstoneProjectName}\par
			\vfill
            {\large Prepared for}\par
            \Huge \CapstoneSponsorCompany\par
            \vspace{5pt}
            {\Large\NameSigPair{\CapstoneSponsorPerson}\par}
            {\large Prepared by }\par
            Group\CapstoneTeamNumber\par
            % 5. comment out the line below this one if you do not wish to name your team
            \CapstoneTeamName\par
            \vspace{5pt}
			{\Large
                \NameSigPair{\GroupMemberOne}\par
                \NameSigPair{\GroupMemberTwo}\par
                \NameSigPair{\GroupMemberThree}\par
            }
            \vspace{20pt}
        }
        \begin{abstract}
        In this document we describe everything we have done in our project to create a deep learning program that can learn to play the game Galaga.
	This project was created to be used as a training tool for NVIDIA's Deep Learning Institute.
	In this document we go over in a bit more detail the purpose of the project.
	We also talk about the progress we made and the problems we faced throughout development.


        \end{abstract}
    \end{singlespace}
\end{titlepage}

\newpage
\pagenumbering{arabic}
\tableofcontents

\section{Introduction}
The main goal of our project was to create a neural network that can learn to play the game Galaga.
During the research phase we set up some benchmarks for how well the neural net should play at the project's completion.
There were a few requirements for the neural network.
The neural network must run on NVIDIA's Jetson developer kit.
It will be used by trainers and students in the field of deep learning.
NVIDIA's Deep Learning Institute will take the work that we do and use it to create a learning course.
It must be possible to recreate this project, or it will not make a good learning tool.
As such, our documentation had to be detailed enough to give readers a clear understanding of how we made the system.
With modern technology and hardware becoming more and more advanced new computational techniques are becoming possible.
The field of AI has improved dramatically is the last 10 to 20 years.
Machine learning has become a lot more popular since modern hardware can now support it.
With machine learning now being more prevalent, it's important to help others build a thorough understanding so that computer scientists can take full advantage of this new technology.
Our client was Mark Ebersole, a Training Development Manager and Director at NVIDIA.
The members on the team were Luay Alshawi, Gabe Morey, and Chris Johnson.
Luay was in charge of the decision layer. Gabe handled the communication of commands to the neural net, and Chris created the input data for training.
The client acted as a supervisor, signing off on each step of the process.

\section{Requirements document}

\documentclass[letterpaper,10pt]{article}

\usepackage{graphicx}
\usepackage{amssymb}
\usepackage{amsmath}
\usepackage{amsthm}

\usepackage{alltt}
\usepackage{float}
\usepackage{color}
\usepackage{url}

\usepackage{enumitem}

\usepackage{geometry}

\usepackage{titling}


\geometry{textheight=8.5in, textwidth=6in}

%\usepackage{hyperref}

\def\name{Christopher Johnson, Gabe Morey, Luay Alshawi}

\parindent = 0.0 in
\parskip = 0.1 in

\pretitle{\begin{center}\Huge\bfseries}
\posttitle{\par\end{center}\vskip 0.5em}
\preauthor{\begin{center}\Large\ttfamily}
\postauthor{\end{center}}
\author{Christopher Johnson, Gabe Morey, Luay Alshawi}
\title{Requirements Documents}

\date{October 27, 2016}

\begin{document}

\begin{titlingpage}
\maketitle
CS461 Fall Term

\begin{abstract}
Deep learning is a form of graphics-based machine learning that uses multi-layered neural networks and high-speed 
GPU processing to make sense of images, sound and text. 
The goal of this project is to create an application which uses deep learning to perform a particular task and run on the Jetson TX1 Developer Kit.
The application will be utilized as part of a training program to teach users how to solve similar problems with deep learning. NVIDIA's Deep Learning Institute has a high demand for training in this field of artificial intelligence. 
This project will provide a tool for NVIDIA's training purposes. 
Because of this the project must also run on their develper kit.
Our team will be using version control to work efficiently and keep objectives organized. 
We will have our repository stored on Git Hub and use issue tracking and status reports to keep track of jobs.
\end{abstract}

\end{titlingpage}
Our team name will be AI Gaming.
Our team will be devoting at least ten hours a week to our project.
Our solution will be implemented in python and use openCV.

The project must use a deep learning neural net to accomplish a task.
The machine will take video input and use it to learn how to play the game Galaga. 
We will need to optain two copies of the game.
A camera or screen capture device will be needed to retrieve this input. 
The neural net must be trained and allowed to create its own code to accomplish the task.

The project will not have hardcoded game interaction other than what is necessary to give the neural net control of the game.

The project will have to be able to run in the Jetson TX1. 
This will also require extra Jetson specific hardware.
We will also need a second computer to run the game while the machine watches and plays.

The solution of our project will need to be able to beat the first level of the game at least 90\% of the time. 
Ideally the system should be able to make it past level five approximately 50\% of the time.
The ability of the system to make it farther than that on average, or to increase the percent chance on it making it to certain points, will be considered a stretch goal.



\end{document}
\grid


\section{Changes made from the requirements document}

\section{Design document}

%Copyright 2014 Jean-Philippe Eisenbarth
%This program is free software: you can
%redistribute it and/or modify it under the terms of the GNU General Public
%License as published by the Free Software Foundation, either version 3 of the
%License, or {at your option} any later version.
%This program is distributed in the hope that it will be useful,but WITHOUT ANY
%WARRANTY; without even the implied warranty of MERCHANTABILITY or FITNESS FOR A
%PARTICULAR PURPOSE. See the GNU General Public License for more details.
%You should have received a copy of the GNU General Public License along with
%this program.  If not, see <http://www.gnu.org/licenses/>.

%Based on the code of Yiannis Lazarides
%http://tex.stackexchange.com/questions/42602/software-requirements-specification-with-latex
%http://tex.stackexchange.com/users/963/yiannis-lazarides
%Also based on the template of Karl E. Wiegers
%http://www.se.rit.edu/~emad/teaching/slides/srs_template_sep14.pdf
%http://karlwiegers.com
% \documentclass[letterpaper,10pt]{article}
\documentclass{scrreprt}
\usepackage{listings}
\usepackage{underscore}
\usepackage[bookmarks=true]{hyperref}
\usepackage[utf8]{inputenc}
\usepackage[english]{babel}
\usepackage{graphicx}
\usepackage{amssymb}
\usepackage{amsmath}
\usepackage{amsthm}
\usepackage{afterpage}

\usepackage{alltt}
\usepackage{float}
\usepackage{color}
\usepackage{url}

\usepackage{enumitem}

\usepackage{geometry}

\usepackage{titling}
\usepackage{rotating}
\usepackage{pgfgantt}
\usepackage{graphicx}
\usepackage{xcolor}
\usepackage{anyfontsize}

\ganttset{group/.append style={orange},
milestone/.append style={red},
progress label node anchor/.append style={text=red}}

\newcommand\blankpage{%
    \null
    \thispagestyle{empty}%
    \addtocounter{page}{-1}%
    \newpage}

\geometry{textheight=8.5in, textwidth=6in}
\hypersetup{
    bookmarks=false,    % show bookmarks bar?
    pdftitle={Software Requirement Specification},    % title
    pdfauthor={Jean-Philippe Eisenbarth},                     % author
    pdfsubject={TeX and LaTeX},                        % subject of the document
    pdfkeywords={TeX, LaTeX, graphics, images}, % list of keywords
    colorlinks=true,       % false: boxed links; true: colored links
    linkcolor=blue,       % color of internal links
    citecolor=black,       % color of links to bibliography
    filecolor=black,        % color of file links
    urlcolor=purple,        % color of external links
    linktoc=page            % only page is linked
}%
\def\myversion{1.0 }
\date{}
%\title
\usepackage{hyperref}
\begin{document}

\begin{flushright}
    \rule{16cm}{5pt}\vskip1cm
    \begin{bfseries}
        \Huge{Design Document}\\
        \vspace{1.9cm}
        for\\
        \vspace{1.9cm}
        Deep Learning on Embedded Platform\\
        \vspace{1.9cm}
        \LARGE{Version \myversion}\\
        \vspace{1.9cm}
        Prepared by Christopher Johnson, Luay Alshawi, Gabe Morey\\
        \vspace{1.9cm}
        CS 461 Fall 2016\\
        \vspace{1.9cm}
        \today\\
    \end{bfseries}
\end{flushright}
\afterpage{\blankpage}

\tableofcontents

\chapter{Overview}
\section{Scope}

\section{Purpose}

\section{Intended Audience}

\section{Conformance}

\chapter{Definitions}

Deep Learning or Machine Learning - Using
API - Application Programming Interface

\chapter{Conceptual Model for Software Design Descritions}

\section{Software Design in Context}

\section{Siftware Design Description within the life cycle}

\subsection{Influences on SDD preparation}

\subsection{Influences on software life cycle products}

\subsection{Design verification and design role in validation}

\chapter{Design descrition information content}

\section{Introducation}

\section{SSD identification}

\section{Design stakeholders and their concerns}

\section{Design views}

\section{Design viewpoints}

\section{Design elements}

\subsection{Design entities}

\subsection{Design attributes}

\subsection{Name attribute}

\subsection{Type attribute}

\subsection{Pupose attribute}

\subsection{Author Attribute}

\subsection{Design relationships}

\subsection{Design constraints}

\section{Design overlays}

\section{Design rationale}

\section{Design language}

\chapter{Design viewpoints}

\section{Introducation}

\section{Context viewpoint}

\section{Composition viewpoint}

\section{Logical viewpoint}

\section{information viewpoint}

\section{Pattns use viewpoint}

\section{Interface viewpoint}

\section{Structure viewpoint}

\section{Interaction viewpoint}

\section{State dynamics viewpoint}

\section{Algorithm viewpoint}

\section{Resource viewpoint}




\end{document}

\section{Changes made from the design document}

\section{Technology review}

%Copyright 2014 Jean-Philippe Eisenbarth
%This program is free software: you can
%redistribute it and/or modify it under the terms of the GNU General Public
%License as published by the Free Software Foundation, either version 3 of the
%License, or (at your option) any later version.
%This program is distributed in the hope that it will be useful,but WITHOUT ANY
%WARRANTY; without even the implied warranty of MERCHANTABILITY or FITNESS FOR A
%PARTICULAR PURPOSE. See the GNU General Public License for more details.
%You should have received a copy of the GNU General Public License along with
%this program.  If not, see <http://www.gnu.org/licenses/>.

%Based on the code of Yiannis Lazarides
%http://tex.stackexchange.com/questions/42602/software-requirements-specification-with-latex
%http://tex.stackexchange.com/users/963/yiannis-lazarides
%Also based on the template of Karl E. Wiegers
%http://www.se.rit.edu/~emad/teaching/slides/srs_template_sep14.pdf
%http://karlwiegers.com
% \documentclass[letterpaper,10pt]{article}
\documentclass{scrreprt}
\usepackage{listings}
\usepackage{underscore}
\usepackage[bookmarks=true]{hyperref}
\usepackage[utf8]{inputenc}
\usepackage[english]{babel}
\usepackage{graphicx}
\usepackage{amssymb}
\usepackage{amsmath}
\usepackage{amsthm}

\usepackage{alltt}
\usepackage{float}
\usepackage{color}
\usepackage{url}

\usepackage{enumitem}

\usepackage{geometry}

\usepackage{titling}
\usepackage{rotating}
\usepackage{pgfgantt}
\usepackage{graphicx}
\usepackage{xcolor}
\usepackage{anyfontsize}

\ganttset{group/.append style={orange},
milestone/.append style={red},
progress label node anchor/.append style={text=red}}

\geometry{textheight=8.5in, textwidth=6in}
\hypersetup{
    bookmarks=false,    % show bookmarks bar?
    pdftitle={Software Requirement Specification},    % title
    pdfauthor={Jean-Philippe Eisenbarth},                     % author
    pdfsubject={TeX and LaTeX},                        % subject of the document
    pdfkeywords={TeX, LaTeX, graphics, images}, % list of keywords
    colorlinks=true,       % false: boxed links; true: colored links
    linkcolor=blue,       % color of internal links
    citecolor=black,       % color of links to bibliography
    filecolor=black,        % color of file links
    urlcolor=purple,        % color of external links
    linktoc=page            % only page is linked
}%
\def\myversion{1.01 }
\date{}
%\title
\usepackage{hyperref}
\begin{document}

\begin{flushright}
    \rule{16cm}{5pt}\vskip1cm
    \begin{bfseries}
        \Huge{Technology Review and Implementation Plan}\\
        \vspace{1.9cm}
        for\\
        \vspace{1.9cm}
        Deep Learning on Embedded Platform\\
        \vspace{1.9cm}
        \LARGE{Version \myversion}\\
        \vspace{1.9cm}
        Prepared by Christopher Johnson, Luay Alshawi, Gabe Morey\\
        \vspace{1.9cm}
        CS 461 Fall Term\\
        \vspace{1.9cm}
        \today\\
    \end{bfseries}
\end{flushright}



The role of making sure everything runs smoothly on the platform required of us to implement out solution on can be devided into three categories.
These categories are finding the right communication methods with our client for group meetings, deciding what camera technology we need, and figuring out the best way to visialize the data.
For each of these three tasks I have found three approaches that I have research and given thorough consideration.
This is to ensure that we are making the best possible decisions when working with the Jetson TX1.

There are many very popular and suitable tools available to conduct online meetings.
These include Skype, Discord, and Team Speak.
Skype was founded in 2003 by Niklas Zennström.
Skype has a lot of features that Discord and Team Speak don't have.
One of these features includes screen sharing.
This feature could be very beneficial during group meetings when a group member has to explain something visually.
Our group has used Skype twice in the past and the sound quality was good enough for everyone to understand each other.
Discord and Team Speak also do not allow video calls, they only provide groups to voice and text chat.
One down side with Skype however is the presence of bugs.
Among reviewers, skype is the least reliable.

Discord, which is the most recent VoIP service released, is known for its very nice user interface.
This service also does not require a download.
Users can log on and join a Discord server using their browser.
Discord hosts its own servers which are very reliable.
Call quality is not limited to a group members poor internet.
Discord is also very light weight and puts much less strain on the cpu than Skype and Team Speak.
However, Discord does not have many of the Skype features including video calling and screen sharing.

Team Speak is known for its sound quality.
Team Speak is the second oldest voice chat service, which has been in development since 2004. 
The newest version is Team Speak 3.
Team Speak has the greatest sound quality and reliability out of all three VoIP applications.
One of the big problems with Team Speak however is its user interface.
For beginner users it can be a lot more difficult to figure out.
Many reviewers say that it has the worst user interface design compared to Discord and Skype.
This is probably a product of its customizability.
There are many ways to improve a users experience with Team Speak including downloading user interface package downloads.

For deciding what kind of camera we should use, there are a lot of factors to consider.
The camera will need to be able to record in HD with at least 30 frames per second.
The program will need to be able to make out shapes for its visual computing.
Leopard Imaging had partnered with NVIDIA to build cameras spceifically for the Jetson TX1.
This camera meets the miniumum requirements and was built specifically for machine vision.

There is also another camera offered by Leopard Imaging that shoots at 30 frames per second and has 4k resolution.
This camera is the same price as the one HD video camera.
Both are \$400.
A better resolution will allow our program to better understand what it is seeing and be able to compute accordingly.

There is also a camera that our client is supplying us.
This is likely the camera best used since it is provided freely and meets the needs of our project.

The third task involves figuring out how to visualize the data gathered from the program.
There are many ways in which this could be done.
We could have live data visualization, that shows the time, number of lives, number of near misses, and other such useful information to help gage how well the program is performing during the current playthrough.
We could have live data visualization that is also graphical and includes graphs and figures.
We could also have the data only appear at the end after the program has completed so that only the end result is displayed.
Live data visualization can offer a few things for data analysis.
In the beginning this might be very useful because during the start of the project the program may only survive a few seconds of the game.
Later however, when the program is beating more complicated levels, data gathered during the playthough could indicate areas where the program struggles.

Live data that can be displayed visually could enhance the benefits of seeing data gathered thoughout the playthrough.
Graphs that recorded which level the program lost a life at and what was the thing that hit it could easily be displayed visually to enhance comprehension.
One problem with live visualization is that it could also flood the screen with too much information, especially when the playthroughs get really long.

One benefit to having a smaller amount of data appear at the end giving a summary of the playthrough is the simplicity.
An end of game summary could previde more of the information we need and leave out a lot of unnecessary items.
A lot of the live information can also be gathered by just watching the playthrough.
We'll be able to see when the program lost a life or suddenly could not compute how to avoid objects.





\end{document}

\section{Changes made from the technology review}

\section{Weekly blog posts}

\subsection{Fall Week 3}
Our first weekly progress report was for week three.
We had been planning on brainstorming ideas and figuring out exactly what we wanted to do for our project.
Our client gave us a pretty good description of what we were doing and some examples of projects that revolved around deep learning.
Our progress as of this point had involved getting a better understanding of our problem.
Our main issue was setting up meetings that worked for everyone's schedules.
We also started working on the problem statement where each person on the team was assigned to do a portion of the document.
Gabe was responsible for Problem definition, Chris was responsible for proposed solution and Luay for the performance metrics.

\subsection{Fall Week 4}
On week four the team met two times to discuss our problem statement's document and to improve it, also, we met with our TA Vee for the first time.
Also, during our meeting, we decided what our project is going to be.
Our project will use deep learning to train a machine to identify mesquite and point at them using a laser.
The plan for next week is to gather as many requirements regarding the project to make progress and we planned to discuss ideas with our client to get a better idea of what each idea would require.

\subsection{Fall Week 5}
This week we were able to get the problem statement submitted, turn in a rough draft of our requirements document, and schedule another meeting with our client.
There was still a little uncertainty with the project we were going to do and the things that we needed for it but we ended up settling on a game playing project.
This project would learn how to play the game galaga by looking at gameplay.
After that we were able to get a rough draft for our requirements document.
We planned to meet again with our client via skype and hash out some more of the specifics.

\subsection{Fall Week 6}
This week we still had trouble catching up.
We needed work revising some of our previous documents now that we had a better idea of what we were doing.
We also met with Kirsten to go over what we had on the requirements document where she helped us on the formatting and gave us feedback.
We also went over the problem statements and did the changes based on the feedback we received from Kirsten.
Currently, we are waiting for our Client to sign the requirements document.
But, we still had a bit of trouble coordinating since all of our schedules are pretty busy and our client's schedule even more busy.
Next week we thought we'd be able to get our requirements document in and get started on our tech review.

\subsection{Fall Week 7}
This week the team met to revise the requirements document.
The necessary revisions were made based on the feedback we have received from our client.
Also, each member of the team knows what to do for the tech review document.
However, we didn't get a whole lot done this week, but we were able to get a signature back from out client for our requirements document and gave it to our TA who was able to give us some ideas to improve it.
It was a pretty busy week. We plan to add improvements to our requirements document and turn it in again before starting on our tech review.

\subsection{Fall Week 8}
This week we finished the Tech Review and tried to start on the design document.
The tech review ran into trouble and we couldn't quite finish before it was due.
We did send in for an extension and luckily were not the only ones.
We managed to turn it in complete by the new cutoff date discussed in class.
With that done, we set our sights on the design document and met Friday to get started.
We reviewed the IEEE standard and tried to brainstorm a way to approach it, but ultimately didn't get much of the document written.
We split off to work on relevant sections by ourselves and reconvene later.

\subsection{Fall Week 9}
This was a short week because of thanksgiving break.
We met up and discussed briefly our design document and tried to hash out roughly what our design was going to look like.
We weren't able to get the design document in by Wednesday like we had hoped.
We did a lot of research but still weren't feeling very confident about our understanding of our design.
We still had a lot of research to do and were hoping to have made a good dent on the design document by Sunday.

\subsection{Fall Week 10}
This week was pretty stressful.
We tried to get the design document done as best we could despite still being fuzzy about how the design was supposed to go.
We planned on arranging a meeting with people who can better explain details of neural nets so we have a better idea of the best approach.
We sent Mark a rough design document before getting help from the professor on how to improve it.
We sent a new copy of the document on Thursday and another on Friday.
On Friday we spent a good portion of the day working on our project report.

\subsection{Winter Break}
Winter break was an overwhelmingly productive period for this project.
Without classes looming over our heads, we had time to more thoroughly research our project and vastly improve the design of our neural network.
We found out how to properly set up our neural network parameters through Caffe and SSD as well as how to properly set up images for the training.
By the end of this period, we had the beginnings of our neural network and knew how to train it.

\subsection{Winter Week 1}
This week we met in order to run and test the jetson kit we received from our client.
We were hoping to install Caffe and test the Single Shot Detection neural network. 
We were able to run ubuntu on it. 
Then we started to discuss what we currently have and how the training was done using SSD. 
We realized we were doing something that could be wrong because we are supposed to train the neural network to play the game and not just be able to identify the game objects. 
So we emailed our Client and ask whether what we are doing is fine. 
We also discussed other methods of approach we could use to complete the project.

\subsection{Winter Week 2}
This week we heard back from our client and got in touch with an engineer from NVIDIA. 
He answered some of the questions we'd been having and we felt reaffirmed in our initial approach using Caffe.
We decided that sticking with Caffe was the best decision and decided to implement a custom layer to the neural network to allow it to make in-game decisions.

\subsection{Winter Week 3}
We had an important meeting this week to delegate tasks so we could get our system running. 
Chris was placed in charge of all things hardware, both acquiring things and making them work in the way we needed them to.
Luay was placed in charge of implementing the decision layer of our neural network so that our training could move on to the next level.
Gabe was placed in charge of creating a communications protocol to allow the neural network to send its commands to the game, which is running ona seperate computer.
Everybody took on the responsibility of gathering and labeling images to use in training.

\subsection{Winter Week 4}
In a meeting this week we decided who would take on the extra duties of team leader. 
Chris offered to take on the role, and so set about creating the OneNote setup we need moving forward.
Luay ran our first major training sprint, which took 13 hours to complete using 66 images. 
The lack of high speed is due to lacking availability of super high-powered gpu computing. 
Results were promising, though still not precise as we'd like.
Gabe began looking into different ways to set up the communication and found that the emulator had inbuilt functionality for Lua scripting, including its own library of direct control functions.

\subsection{Winter Week 5}
This week we continued work on our indivividual segments. 
For the comm script, codebase was changed from Lua to Python in order to greate a more universally applicable solution, though the keypress commands are still operating system specific.
The decision layer was still in the process of being implemented.
A number of images were labeled and added to the training base to try and improve performance.

\subsection{Winter Week 6}
This week was a mad dash to get everything edited and put together for the midterm report due date.
We've finished up the baseline comm code and the decision layer and have begun testing the system as a whole, with the full hardware setup put together.
We also began editing our old documents with new and relevant information and prepared the midterm report for turn-in.

\subsection{Winter Week 7}
This week we did present our project to our TA and show a demo of the Neural network plays the game.
We worked on improving the decision algorithm.
The original algorithm was mainly developed to make decisions based on defense.
That being said the neural network only makes decisions to make the player survive throughout the game.
However, the new approach is to stay away from enemies as well as move the player the enemies’ direction and shoot them.
Hopefully, by making decisions that can attack enemies and stay away from them will make the neural network reaches higher levels.

\subsection{Winter Week 8}
Work was done to improve the training and image detection.
Unfortunately, the results were the same as before and no improvements.
It seems we need to gather more annotated images for training.

\subsection{Winter Week 9}
This week we sent our midterm report to our client for a signature.
We currently have 160 images for training.
The goal is to reach around 500 before EXPO.
We're able to run the game at 10 frames per second now.

\subsection{Winter Week 10}
We met up this week to touch base and get a rough draft of our EXPO poster.
We each recorded our portions of the final presentation and stitched them together.
Commands are successfully being sent to the neural net and the decision layer has seen improvement.
We can still only have the neural net play the game at 10 frames per second.

\subsection{Spring Week 1}
This week we had a lecture to prepare us for the coming months.
We have only training left to do before expo so this week we started creating annotations for the neural net's image detection.

\subsection{Spring Week 2}
This week I was able to get some annotations done and we found a group to discuss our poster with.
We're feeling pretty relaxed now that most of the work is done.
This week we were able to find a group to discuss our poster with for the extra credit.
Luay started working on improving the defense and attack algorithm.
Gabe decided to work on trying to run tensorRT on the Jetson TX1 and try to integrate it with SSD.
The plan for next week is to have work done on the decision layer algorithm and test it on the Jetson TX1 as well as home computer GPU.

\subsection{Spring Week 3}
Chris made more annotations.
We still have yet to meet with our team to discuss our poster.
We need to make a final draft soon.
We also have to get it to Mark as soon as possible since it will probably take him a while to get it signed.

\subsection{Spring Week 4}
This week we met up and figured out what we had left to do before the code freeze on May 1st.
We realized there was a lot on the requirements document that we still needed to update and we weren't sure if we could meet a lot of them by May 1st.
The neural net still isn't performing as well as it should be.
We fixed up our poster, got our group photo, and sent the poster to Mark and the professors.
We also asked the professors how much we needed to fix with the requirements document.
\newline\newline
Luay was given a Jetson TX2 to test our project on.
The goal was to compare the performance between both the TX1 and TX2.
Due to the differences in GPU architecture on both boards the GPU performance of Caffe was slow.
This issue was solved by altering the makefile configuration.
The maximum performance with the TX1 that we could get was 10 FPS using our custom nerual network model.
On the other hand, Gabe was working on improving the decision algorithm on Jetson Tx1 and tensorRT.
There were some difficulties finding resources on integrating tensorRT.
Luay tried working on TensorRT as well as improving the decision layer.
Writing a tensorRT program unfortunately wouldn't work because tennsorRT does not support custom Caffe layers which was stated on the release note and was the complaint error when running the program.
\newline\newline
By the end of the week the neural network played the game winning stage 1 90% of the time.
The highest score reached so far is 15520 which is a new record.
Also, the neural network was able to reach stage 4 of the game and this is the best result so far.
We are hoping to do a live demo during expo by having one Laptop that has the game and the Jetson TX1.
By using the camera on board we are going to save space and be able to do a live demo.

\subsection{Spring Week 5}
This week we got through the code freeze after making sure everything we used was in our repo.
We also finished our poster and submitted it for printing.
Luay recently was able to boost the performance of the neural net using screen capturing instead of video recording.
We will most likely continue working up until Expo on May 19th.
There are still two more lectures before then.
\newline\newline
Luay continued working on installing OpenCV 3.2.0 to use the Jetson TX1 on board camera.
The onboard camera is better quality and capable of capturing full HD compare to the current USB camera which is only able to capture up to 720p.
The process of installing OpenCV 3.2.0 was painful because there was not enough disk space and had to spend the time to clean unnecessary files.
Although, the process of installing OpenCV from source code took more than an hour due to the low power nature of the Jetson TX1.
By using the onboard camera we might be able to do a live demo during the expo.
Finally, we are planning to email our client this week to give him updates on our current progress.

\subsection{Spring Week 6}
This week we did a lot of important things to prep for expo.
The first was meeting with another group and Professor Winters on Tuesaday to discuss our pitches and poster.
We worked on how we'd talk to people on the expo floor and got a lot of useful insight from the professor and our peers.
We also got to work on the midterm report and presentation and divided out duties for how to tackle it.
With everything planned out, we will be finishing off the midterm and getting prepped for expo on Friday the 19th.

\section{Code Sample}
\subsection{Python Script Plays the Game}
\begin{lstlisting}[language=Python,caption=Python script captures image and feed to nerual network to send a decision to the other computer.]
while True:
    flag, frame = cap.read() # Capture image from the webcam
    start = time.time()
    if not flag:
        continue
    numerate = 0
    image = frame
    transformed_image = transformer.preprocess('data', image)
    net.blobs['data'].data[...] = transformed_image # input the image to neural network
    # Forward pass.
    decision = net.forward()['decision']# Get decision from the neural network.
    if(decision[0]==0.0): #send right command to move away from enemies
        print "right"
        payload = {'keys': 'x'}
        response = unirest.post(url, params=json.dumps(payload), headers=headers, callback=callback_function)
        print response
    elif decision[0]==1.0: #send left command to move away from enemies
        print "left"
        payload = {'keys': 'z'}
        response = unirest.post(url, params=json.dumps(payload), headers=headers, callback=callback_function)
        print response
    elif decision[0]==2.0: #send shoot command
        print "shoot"
        payload = {'keys': "f"}
        response = unirest.post(url, params=json.dumps(payload), headers=headers, callback=callback_function)
        print response
    elif decision[0]==10.0: #send right command to find enemies to shoot
        print "right _x"
        payload = {'keys': "_x"}
        response = unirest.post(url, params=json.dumps(payload), headers=headers)#, callback=callback_function)
        print response
    elif decision[0]==100.0:  #send left command to find enemies to shoot
        print "left _z"
        payload = {'keys': "_z"}
        response = unirest.post(url, params=json.dumps(payload), headers=headers)#, callback=callback_function)
        print response
    cv2.imshow('video', frame)
    cv2.waitKey(1)
\end{lstlisting}
\subsection{Nodejs Server}
\begin{lstlisting}[language=VBScript,caption=Nodejs script runs a server and listen for client request to execute keyboard commands.]
	var express = require('express')
	var app = express()
	var robot = require('robotjs');
	var robotjs = require('robot-js');
	var bodyParser = require('body-parser')
	var keyboard = robotjs.Keyboard();

	app.use(bodyParser.json())

	app.listen(5000, function () {
	  console.log('app listening on port 5000!')
	})
	app.post('/api/sendkeys', function (req, res) {
	var key = req.body.keys;
	res.json({ success: 'message'});
	  if (key=="z")
	  {
	    keyboard.press(robotjs.KEY_Z);
	    robotjs.Timer.sleep (160);
	    keyboard.click(robotjs.KEY_D);
	    keyboard.release(robotjs.KEY_Z);
	  }
	  else if (key=="_z")
	  {

	    keyboard.press(robotjs.KEY_Z);
	    robotjs.Timer.sleep (25);
	    keyboard.click(robotjs.KEY_D);
	    keyboard.release(robotjs.KEY_Z);
	  }
	  else if(key=="x")
	  {
	    keyboard.press(robotjs.KEY_X);
	    robotjs.Timer.sleep (160);
	    keyboard.click(robotjs.KEY_D);
	    keyboard.release(robotjs.KEY_X);
	  }
	  else if(key=="_x")
	  {
	    keyboard.press(robotjs.KEY_X);
	    robotjs.Timer.sleep (25);
	    keyboard.click(robotjs.KEY_D);
	    keyboard.release(robotjs.KEY_X);
	  }
	  else if(key=="f")
	  {
	    keyboard.click(robotjs.KEY_D);

	  }
})
\end{lstlisting}
\section{Project Photos}
\subsection{Detection Layer Result Sample 1}
\begin{center}
  \makebox[\textwidth]{\includegraphics[width=\paperwidth]{d1.eps}}
\end{center}
\subsection{Detection Layer Result Sample 2}
\begin{center}
  \makebox[\textwidth]{\includegraphics[width=\paperwidth]{d2.eps}}
\end{center}
\subsection{Training plot Loss v.s Accuracy}
\begin{center}
  \makebox[\textwidth]{\includegraphics[width=\paperwidth]{plot.eps}}
\end{center}
\subsection{Final Project Setup}
\begin{center}
  \makebox[\textwidth]{\includegraphics[width=\paperwidth]{setup.eps}}
\end{center}





\section{Current Progress}
Throughout the first term we laid out all the ground work for out project.
We've created documentation for our project requirements, our decisions for the technology that we want to use, and how we will be using that technology in our project.
These documents have been given to our client who has signed and returned them to ensure that everything can be smoothly recreated and used in an educational setting.
\newline\newline
Over winter break we began implementation.
The project at present has the following components: a neural network still in the stages of training to properly identify in game objects, a set of python scripts implementing port communication protocol to allow the neural network to converse with the system hosting the game, and a hardware setup with camera to allow the Jetson to see what's happening on the other computer.
Though there are many bugs to iron out, the system can generate actions based on the situation of the player, and commands can be given to the game from the neural network.
Moving forward the main goals are increasing the speed of command, the accuracy of gameplay, and ironing out bugs.

\section{Problems and Solutions}
We ran into several problems throughout the first term.
One problem we had was our busy schedules.
It was difficult to find times to meet up and work on projects.
We had to schedule our weekly TA meeting on Fridays which wasn't ideal.
Our TA is planning on scheduling all his meetings on Monday which might help for Winter term.
We're also handling this problem by improving our lines of communication with each other.
By staying in touch regularly and planning ahead, we have mitigated a lot of the scheduling issues that have plagued this term.
\newline\newline
Another problem we had was getting our documents signed.
Our client was very busy throughout the term and it took a while to to get responses from him.
All of our assignments were either turned in late or without a signature.
He said that his schedule should soon be clearing up and he won't have to travel as much so it shouldn't be as bad next term.
Just in case our client remains difficult to contact we planned to ask him about more methods of contact as well as maybe finding other people at the company who can weigh in when he's unavilable.
\newline\newline
Another problem arose in this project due to the vagueness of the initial project.
Initially, we had no solid idea what our project was going to be, other than that it had to involve a neural network.
Issues contacting our client compounded this issue as we needed information about what kinds of projects would be considered suitble for the task at hand.
Eventually we were able to establish solid enough contact to determine our actual task.
\newline\newline
The final issue first term arose from the difficulty of actually understanding neural networks.
Learning about neural networks required much more research than we initially assumed, and the field is incredibly complex.
In order to better understand what was going on, we had to try and cram tons of research into a very small period.
This met with middling success and we still weren't fully confident in our design.
We did, however, have a solution in mind.
We reached out to our client and made contact with an engineer at NVIDIA.
We also took advantage of the increased free time during winter break to more thoroughly study our subject and prepare the initial system.
\newline\newline
The second term started off with a long winter break, during which major breakthroughs and problems were discovered.
Perhaps the biggest thing to come out of winter break was a new understanding what exactly we had to do to get our neural network functioning.
By the end of the break we had a much more solid plan for what we were going to accomplish.
However, our initial approach to the network required some custom implementation that we weren't sure how to do yet.
We didn't know how to get the neural net telling the game how to play.
To solve this, we spoke with our client and he connected us with an engineer at NVIDIA who had more experience with neural networks.
With his help, we came to the decision of implementing a single decision layer into the neural net that would take the output of the image processing layers and produce an output in the form of a game command.
\newline\newline
The next big problem we needed to approach was the setup we need to use to get the setup correct for playing the game.
Because we need to have the Jetson run the neural network, we've been gathering parts to allow us to put Galaga on a PC seperate from the jetson for playing.
We've purchased a camera to watch the screen of the other PC and send that back as input.
At the moment reflectivity is hampering this process, so we are working on getting a screen with as little reflectiveness as possible.
Another major part of this setup was getting a program setup that could relay the commands from the neural net to the game.
To solve this, we set up a basic communication server setup through python, which waits for commands and then simulates the keypresses ordered through a Windows api.
\newline\newline
The major issue that hung over our heads heading into the end of winter term was the speed the neural network.
On a normal PC, our system actually ran decently, but the Jetson is considerably slower.
With average processing speed between 3 and 9 fps (frames per second) when implemented on the Jetson, our network was simply too slow to play a 60 fps game right now.
We tested ways around this limitation and eventually improved the speed.
\newline\newline
The biggest improvement to the project since the middle of winter term was the frame rate.
We now get around 12-15 frames processed per second with the neural network.
This is possible because of optimizations we found for the Jetson that help divert resources to running a neural network.
We set our sights on improving the decision layer code, as well as getting the neural network trained further.
\newline\newline
In order to get ready for Expo this term, we started with a few goals.
Goal number one was to get our neural network trained better.
Goal number two was to improve the decision layer logic for the network.
Our final goal was to buff up the frame rate of the neural network response.
Along the way we also had to finish the poster for expo and prepare to present.
\newline\newline
Getting the neural network trained was a matter of continuing to do what he'd done for a while now.
We needed to gather more images and annotate them for the training process.
Getting more images was a central part of getting the project ready for expo.
We've managed to build up a good image base with over one hundred annotations.
After the latest training run, our detection results have proved better than ever before.
\newline\newline
When it came to the decision layer logic, we tried quite a few methods.
Our initial logic only allowed the network to move away from incoming enemies or shoot them.
After a bit more refinement, we tried a version where the system was allowed to shoot, then move in one command.
This second version had some better performance and its results were used on our expo poster due to the time in term.
Afterwards, we arrived at what is now the final and best version of the decision layer.
This version moves much more often and can prepare to fire when it sees no enemies.
This version also tends to shoot whenever it moves.
This combination helps it find enemy targets as well as shoot reflexively at oncoming enemies.
With this version, we've come to meet our goal of beating the first level of Galaga 90\% of the time.
\newline\newline
The framerate is the goal we've consistently had the most trouble meeting.
While framerate was not a part of our requirements, we wanted to have a reasonable play speed to make the network more exciting.
However, we've never gotten much more than 15 FPS out of the Jetson. 
Pushing the game speed up higher results in the Jetson responding too slow to perform well.
We tried using some optimization tools from NVIDIA, but they didn't properly support the SSD basis of our neural network or its custom layers.
We have signed up for the test of the new version, so we may yet be able to do something.
\newline\newline
One major issue we had to resolve this term was getting results for our poster.
When we first handed it off to our client, he liked what we had but wanted more concrete performance results from our neural network.
The week before the poster's due date became a challenging rush to optimize our system and get good results to put on the poster.
Ultimately, the poster turned out good and the results were usable. 
We do wish we could have had more time, however, as we've since achieved much better results, with a maximum score of 18230 and now being able to regularly beat level 1.
\newline\newline
Overall, we're very happy with the progress and outcome of this project, but we aren't done yet.
We continue to try and optimize it, and we have a lot of documentation to catch up on.
We're a far cry away from where we were at the start of the term, and we're better programmers for it.
That does mean, however, that most of the documentation written then does not match what our project has come to be.
While we do have some edits done from last term, we still have a lot to catch up on if we are to meet the requirements of the class.
It's also important that our documentation be excellent because NVIDIA needs to be able to recreate it as a Deep Learning Institute lab.
With that in mind, we have a lot of work cut out for us to finish this class.






\end{document}
