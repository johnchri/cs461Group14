% \documentclass[letterpaper,10pt]{article}
\documentclass{scrreprt}

\usepackage{graphicx}
\usepackage{amssymb}
\usepackage{amsmath}
\usepackage{amsthm}

\usepackage{alltt}
\usepackage{float}
\usepackage{color}
\usepackage{url}

\usepackage{enumitem}

\usepackage{geometry}

\usepackage{titling}


\geometry{textheight=8.5in, textwidth=6in}

%\usepackage{hyperref}

\def\name{Christopher Johnson, Gabe Morey, Luay Alshawi}

\parindent = 0.0 in
\parskip = 0.1 in

\pretitle{\begin{center}\Huge\bfseries}
\posttitle{\par\end{center}\vskip 0.5em}
\preauthor{\begin{center}\Large\ttfamily}
\postauthor{\end{center}}
\author{Christopher Johnson, Gabe Morey, Luay Alshawi}
\title{Deep Learning on Embedded Platform}
\date{October 19, 2016}

\begin{document}

\begin{flushright}
    \rule{16cm}{5pt}\vskip1cm
    \begin{bfseries}
        \Huge{Problem Statement}\\
        \vspace{1.9cm}
        for\\
        \vspace{1.9cm}
        Deep Learning on Embedded Platform\\
        \vspace{1.9cm}
        Prepared by Christopher Johnson, Luay Alshawi, Gabe Morey\\
        \vspace{1.9cm}
        CS 461\\
    \end{bfseries}
\end{flushright}

\tableofcontents

\chapter{Abstract}
NVIDIA's Deep Learning Institute has a high demand for training programs in the field of artificial intelligence(AI).
Deep learning is a form of graphics-based machine learning that uses multi-layered neural networks and high-speed GPU processing to make sense of images, sound and text, which allows a computer based AI to learn over time.
The goal of this project is to create a system which uses deep learning to learn how to play the game Galaga and runs on the Jetson TX1 Developer Kit.
The results of our project will be utilized as part of a training program to teach users how to solve similar problems with deep learning.
We will not be creating the learning module itself, but our project will contribute to deeper understanding and greater chances for learning in the field of deep learning.

\chapter{Problem Statement}

\section{Problem Definition}
NVIDIA’s Deep Learning Institute is a set of online courses and workshops with the aim of teaching developers how to use and apply deep learning to their own problems.
With this goal in mind, NVIDIA is constantly looking for new and better ways to teach about the subject, as well as further this field of research.
Not many people know what deep learning is, and the concepts of it can seem daunting to some.
It is vital for the field's growth that more people learn how and why to use it.
That's where our project comes in.
Our group’s task is to train a deep learning network to build a system of code that allows it to accomplish a specific task.
Our project needs to be recreatable so that it can contribute to teaching more people about the growing field of deep learning.
When the project is complete, it will be used by Nvidia to make a new course for the Deep Learning Institute.
This course will provide an easy, hassle free way to learn how to train and use deep learning to accomplish tasks.


Deep learning is a rapidly growing field of machine learning that aims to use GPU based neural networks to accomplish a wide variety of tasks.
Deep learning uses a system of neural networks to process data from graphs, plots, visual input, or any set of machine readable data to find patterns and learn how to solve the problem the data represents.
Neural networks are a form of machine learning which use a layered structure of connections to make decisions and store information relevant to a problem.
Their structure was inspired by the neurons of the human brain [1].
Each neuron has a trained input weight, which assigns how correct or incorrect that branch may be.
These weights are developed through intensive training of the neural network as it attempts to complete tasks [1].
The neural net uses these weights to create a probability vector that predicts the outcome chance of success.

\section{Proposed solution}
The traditional problem with neural networks has been their incredible computational requirements.
 It takes massive processing power to train and develop a neural network.
 However, this problem has been partially solved by taking advantage of the power of the modern GPU.
 This is the basis for deep learning, and now it is much more feasible to train a network to be proficient at a task.
 Modern deep learning neural nets can be massive and capable of processing huge amounts of data in order to build their neurons.
 Using deep learning to build code for solving a problem means that only minimal time needs to be spent programming.
 Ideally, the deep learning system should be directed to figuring out how to solve the problem with its own code and systems.
 Now that we have a means of training these networks and using them in the field, there is a growing need for people who know how to train and use these networks.
 Making this technology accessible to more developers is the ultimate goal of NVIDIA's Deep Learning Institute and this project.


Our solution to the problem must be a useful learning tool for others.
It should be easily presentable so people can understand the impact of deep learning on the solution.
 It should be a program that can utilize deep learning and provide a comprehensive example of deep learning's power and capabilities.
 Our solution will be to teach a neural network how to play the game Galaga.
 Teaching the system to play Galaga allows us to have a clearly presentable project with clear ways of defining performance and improvement.
 It also adds a layer of fun to the project, making it more interesting to those who see it.
 Ideally the fun aspect will both encourage people to try and replicate it and generate new interest in the field of deep learning and the things it can accomplish.


\section{Performance Metrics}
Version control will be used in this project to keep track of the tasks and measure the performance for each person on the team as well as the performance of the project.
 Tasks will be labeled to three different categories.
 The First label called “Feature” which will be used to label major tasks toward the project.
 The second label is “Enhancement” which will be used to label tasks that enhance major tasks such as refactoring particular code or add a feature to a major task.
 The last label is “Bug” which will label tasks that are considered closed but require a fix.
 By using three labels, tracking the project’s performance is easier and efficient.
 Initially, all members of the team will make tasks and label them to solve the given problems on this project, and each person can assign a task to himself.
 Moreover, tasks can be reviewed by any member of the team upon completion to ensure the quality and completion of the task.


Finally, related tasks shall be completed using iterations.
 The length of each iteration is going to be two up to three weeks long, and each iteration must solve an induvial problem.
 After an iteration is complete, a review session is going to be held with all team members to discuss progress and test all tasks after merging them together.
 Optionally, the project’s sponsor can join the meeting to review the team's progress.

The measurement of success on this project depends on how well the neural network is trained to play the game Galaga.
The aim is to play the game Galaga and have a winning rate of 90\'\% in the first level.
Ideally the system should be able to make it past level five approximately 50\'\% of the time.

The measurement of success on this project depends on how well the neural network is trained to play the game Galaga.
The aim is to play the game Galaga and have a winning rate of 90\'\% on the first level.
Ideally, the system should be able to make it past level five approximately 50\'\% of the time.
On the other hand, another success is when the team is able to return easily followed documentation to allow a deep learning trainee and trainer to rebuild the project.

\chapter{Refrences}
\section{Bibliography}
[1]N. C. L. Information, /`/`Deep learning,'' NVIDIA Developer, 2016. [Online]. Available: https://developer.nvidia.com/deep-learning. Accessed: Oct. 18, 2016.

\end{document}
\grid
