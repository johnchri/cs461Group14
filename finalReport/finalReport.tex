\documentclass[onecolumn, draftclsnofoot,10pt, compsoc]{IEEEtran}
\usepackage{graphicx}
\usepackage{url}
\usepackage{setspace}

\usepackage{geometry}
\geometry{textheight=9.5in, textwidth=7in}

% 1. Fill in these details
\def \CapstoneTeamName{		Deep Learning on Embedded Platform}
\def \CapstoneTeamNumber{		14}
\def \GroupMemberOne{			Christopher Johnson}
\def \GroupMemberTwo{			Gabe Morey}
\def \GroupMemberThree{			Luay Alshawi}
\def \CapstoneProjectName{		AI Gaming}
\def \CapstoneSponsorCompany{	NVIDIA}
\def \CapstoneSponsorPerson{		Mark Ebersole}

\def \DocType{	%Problem Statement
				%Requirements Document
				%Technology Review
				%Design Document
				Progress Report
				}

\newcommand{\NameSigPair}[1]{\par
\makebox[2.75in][r]{#1} \hfil 	\makebox[3.25in]{\makebox[2.25in]{\hrulefill} \hfill		\makebox[.75in]{\hrulefill}}
\par\vspace{-12pt} \textit{\tiny\noindent
\makebox[2.75in]{} \hfil		\makebox[3.25in]{\makebox[2.25in][r]{Signature} \hfill	\makebox[.75in][r]{Date}}}}

\begin{document}
\begin{titlepage}
    \pagenumbering{gobble}
    \begin{singlespace}
    	% \includegraphics[height=4cm]{coe_v_spot1}
        \hfill
        % 4. If you have a logo, use this includegraphics command to put it on the coversheet.
        %\includegraphics[height=4cm]{CompanyLogo}
        \par\vspace{.2in}
        \centering
        \scshape{
            \huge CS Capstone \DocType \par
            {\large\today}\par
            \vspace{.5in}
            \textbf{\Huge\CapstoneProjectName}\par
            \vfill
            {\large Prepared for}\par
            \Huge \CapstoneSponsorCompany\par
            \vspace{5pt}
            {\Large\NameSigPair{\CapstoneSponsorPerson}\par}
            {\large Prepared by }\par
            Group\CapstoneTeamNumber\par
            % 5. comment out the line below this one if you do not wish to name your team
            \CapstoneTeamName\par
            \vspace{5pt}
            {\Large
                \NameSigPair{\GroupMemberOne}\par
                \NameSigPair{\GroupMemberTwo}\par
                \NameSigPair{\GroupMemberThree}\par
            }
            \vspace{20pt}
        }
        \begin{abstract}
        In this document we describe the design of a deep learning system developed to learn how to play the arcade game Galaga.
		Galaga is an arcade shooter which was released in 1981 by Namco [1].
		This system is designed with the ultimate goal of being turned into a course for the NVIDIA Deep Learning Institute.
		The documentation will outline exactly what hardware will used, how the system will be put together, and what methods will be used in a way that allows others to recreate this project.
        \end{abstract}
    \end{singlespace}
\end{titlepage}

\newpage
\pagenumbering{arabic}
\tableofcontents

\section{Project Purpose}
\section{Current Progress}
\section{Problems and Solutions}
\section{Interesting Code}
\section{Retrospective}

\section{Week by week summary}

\subsction{Week 3}
Our first weekly progress report was for week three.
We had been planning on brainstorming ideas and figuring out exactly what we wanted to do for our project.
Our client gave us a pretty good description of what we were doing and some examples of projects that revolved around deep learning.
Our progress as of this point had involved getting a better understanding of our problem.
Our main issue was setting up meetings that worked for everyone's schedules.

\subsection{Week 4}
Week 4 was a slow week.
We didn't make a whole lot of progress.
We were still unsure of exactly what we wanted to do for our project and were waiting to set up a meeting with our client so we could talk about some of the specifics.
We planned to discuss ideas with our client to get a better idea of what each idea would require.

\subsection{Week 5}
This week we were able to get the problem statement submitted, turn in a rough draft of our requirements document, and schedule another meeting with our client.
There was still a little uncertainty with the project we were going to do and the things that we needed for it but we ended up settling on a game playing project.
This project would learn how to play the game galaga by looking at gameplay.
After that we were able to get a rough draft for our requirements document.
We planned to meet again with our client via skype and hash out some more of the specifics.

\subsection{Week 6}
This week we still had trouble catching up.
We needed work revising some of our previous documents now that we had a better idea of what we were doing.
We still also had a bit of trouble coordinating since all of our schedules are pretty busy and our client's schedule even more busy.
Next week we thought we'd be able to get our requirements document in and get started on our tech review.

\subsection{Week 7}
We didn't get a whole lot done this week.
We were able to get a signature back from out client for our requirements document and gave it to our TA who was able to give us some ideas to improve it.
It was a pretty busy week.
We planned to add improvements to our requirements document and turn it in again before starting on our tech review.

\subsection{Week 8}
Week 8 was another really busy week.
We were able to get our requirements document finished and turned in.
We also got a start on our design document and were hoping to be able to send Mark something by Wednesday, just to stay ahead.

\subsection{Week 9}
This was a short week because of thanksgiving break.
We met up and discussed briefly our design document and tried to hash out roughly what our design was going to look like.
We weren't able to get the design document in by Wednesday like we had hoped.
We did a lot of research but still weren't feeling very confident about our understanding of our design.
We still had a lot of research to do and were hoping to have made a good dent on the design document by Sunday.

\subsection{Week 10}
This week was pretty stressful.
We tried to get the design document done as best we could despite still being fuzzy about how the design was supposed to go.
We planned on arranging a meeting with people who can better explain details of neural nets so we have a better idea of the best approach.
We sent Mark a rough design document before getting help from the professor on how to improve it.
We sent a new copy of the document on Thursday and another on Friday.
On Friday we spent a good portion of the day working on our project report.
We plan on finishing it up a little bit during the weekend and recording our presentation on Monday or Tuesday.

\end{document}
