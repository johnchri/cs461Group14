\documentclass[onecolumn, draftclsnofoot,10pt, compsoc]{IEEEtran}
\usepackage{graphicx}
\usepackage{url}
\usepackage{setspace}

\usepackage{geometry}
\geometry{textheight=9.5in, textwidth=7in}

% 1. Fill in these details
\def \CapstoneTeamName{		Deep Learning on Embedded Platform}
\def \CapstoneTeamNumber{		14}
\def \GroupMemberOne{			Christopher Johnson}
\def \GroupMemberTwo{			Gabe Morey}
\def \GroupMemberThree{			Luay Alshawi}
\def \CapstoneProjectName{		AI Gaming}
\def \CapstoneSponsorCompany{	NVIDIA}
\def \CapstoneSponsorPerson{		Mark Ebersole}

\def \DocType{	%Problem Statement
				%Requirements Document
				%Technology Review
				%Design Document
				Progress Report
				}

\newcommand{\NameSigPair}[1]{\par
\makebox[2.75in][r]{#1} \hfil 	\makebox[3.25in]{\makebox[2.25in]{\hrulefill} \hfill		\makebox[.75in]{\hrulefill}}
\par\vspace{-12pt} \textit{\tiny\noindent
\makebox[2.75in]{} \hfil		\makebox[3.25in]{\makebox[2.25in][r]{Signature} \hfill	\makebox[.75in][r]{Date}}}}

\begin{document}
\begin{titlepage}
    \pagenumbering{gobble}
    \begin{singlespace}
    	% \includegraphics[height=4cm]{coe_v_spot1}
        \hfill
        % 4. If you have a logo, use this includegraphics command to put it on the coversheet.
        %\includegraphics[height=4cm]{CompanyLogo}
        \par\vspace{.2in}
        \centering
        \scshape{
            \huge CS Capstone \DocType \par
            {\large\today}\par
            \vspace{.5in}
            \textbf{\Huge\CapstoneProjectName}\par
            \vfill
            {\large Prepared for}\par
            \Huge \CapstoneSponsorCompany\par
            \vspace{5pt}
            {\Large\NameSigPair{\CapstoneSponsorPerson}\par}
            {\large Prepared by }\par
            Group\CapstoneTeamNumber\par
            % 5. comment out the line below this one if you do not wish to name your team
            \CapstoneTeamName\par
            \vspace{5pt}
            {\Large
                \NameSigPair{\GroupMemberOne}\par
                \NameSigPair{\GroupMemberTwo}\par
                \NameSigPair{\GroupMemberThree}\par
            }
            \vspace{20pt}
        }
        \begin{abstract}
        In this document we describe the design of a deep learning system developed to learn how to play the arcade game Galaga.
		Galaga is an arcade shooter which was released in 1981 by Namco [1].
		This system is designed with the ultimate goal of being turned into a course for the NVIDIA Deep Learning Institute.
		The documentation will outline exactly what hardware will used, how the system will be put together, and what methods will be used in a way that allows others to recreate this project.
        \end{abstract}
    \end{singlespace}
\end{titlepage}

\newpage
\pagenumbering{arabic}
\tableofcontents

\section{Project Purpose}
\section{Current Progress}
\section{Problems and Solutions}
\section{Interesting Code}
\section{Retrospective}


\end{document}
