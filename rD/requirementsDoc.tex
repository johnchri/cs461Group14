\documentclass[letterpaper,10pt]{article}
\usepackage{graphicx}
\usepackage{amssymb}
\usepackage{amsmath}
\usepackage{amsthm}

\usepackage{alltt}
\usepackage{float}
\usepackage{color}
\usepackage{url}

\usepackage{enumitem}

\usepackage{geometry}

\usepackage{titling}
\usepackage{rotating}
\usepackage{pgfgantt}
\usepackage{graphicx}
\usepackage{xcolor}
\usepackage{anyfontsize}

\ganttset{group/.append style={orange},
milestone/.append style={red},
progress label node anchor/.append style={text=red}}

\geometry{textheight=8.5in, textwidth=6in}

%\usepackage{hyperref}

\def\name{Christopher Johnson, Gabe Morey, Luay Alshawi}

\parindent = 0.0 in
\parskip = 0.1 in

\pretitle{\begin{center}\Huge\bfseries}
\posttitle{\par\end{center}\vskip 0.5em}
\preauthor{\begin{center}\Large\ttfamily}
\postauthor{\end{center}}
\author{Christopher Johnson, Gabe Morey, Luay Alshawi}
\title{Requirements Documents}

\date{October 27, 2016}

\begin{document}

\begin{titlingpage}
\maketitle
CS461 Fall Term

\begin{abstract}
Deep learning is a form of graphics-based machine learning that uses multi-layered neural networks and high-speed
GPU processing to make sense of images, sound and text.
The goal of this project is to create an application which uses deep learning to perform a particular task and run on the Jetson TX1 Developer Kit.
The application will be utilized as part of a training program to teach users how to solve similar problems with deep learning. NVIDIA's Deep Learning Institute has a high demand for training in this field of artificial intelligence.
This project will provide a tool for NVIDIA's training purposes.
Because of this the project must also run on their develper kit.
Our team will be using version control to work efficiently and keep objectives organized.
We will have our repository stored on Git Hub and use issue tracking and status reports to keep track of jobs.
\end{abstract}

\end{titlingpage}
Our team name will be AI Gaming.
Our team will be devoting at least ten hours a week to our project.
Our solution will be implemented in python and use openCV.

The project must use a deep learning neural net to accomplish a task.
The machine will take video input and use it to learn how to play the game Galaga.
We will need to optain two copies of the game.
A camera or screen capture device will be needed to retrieve this input.
The neural net must be trained and allowed to create its own code to accomplish the task.

The project will not have hardcoded game interaction other than what is necessary to give the neural net control of the game.
The neural network should define for itself how to respond to in-game stimuli and react accordingly.

The project will have to be able to run in the Jetson TX1.
This will also require extra Jetson specific hardware.
We will also need a second computer to run the game while the machine watches and plays.

The solution of our project will need to be able to beat the first level of the game at least 90\% of the time.
Ideally the system should be able to make it past level five approximately 50\% of the time.
The ability of the system to make it farther than that on average, or to increase the percent chance on it making it to certain points, will be considered a stretch goal.

We must clearly outline our actions taken and resources required at every step of the process.
Our project must be explainable and easy to recreate in order for it to be turned into a lesson plan.
Clear documentation will be provided that outlines how the project was accomplished.

% \begin{turn}{90}
\resizebox{\textwidth}{!}{

%    \centering
     \begin{ganttchart}[%Specs
     y unit title=2cm,
     y unit chart=2cm,
     vgrid,hgrid,
     title height=1,
%     title/.style={fill=none},
     title label font=\bfseries\footnotesize,
     bar/.style={fill=blue},
     bar height=0.7,
%   progress label text={},
     group right shift=0,
     group top shift=0.7,
     group height=.3,
     group peaks width={0.2},
     inline]{1}{50}
    %labels
    \gantttitle{Plan}{50}\\  % title 1
    \gantttitle{W1}{2}
    \gantttitle{W2}{2}
    \gantttitle{W3}{2}
    \gantttitle{W4}{2}
    \gantttitle{W5}{2}
    \gantttitle{W6}{2}
    \gantttitle{W7}{2}
    \gantttitle{W8}{2}
    \gantttitle{W9}{2}
    \gantttitle{W10}{2}
    \gantttitle{W11}{2}
    \gantttitle{W12}{2}
    \gantttitle{W13}{2}
    \gantttitle{W14}{2}
    \gantttitle{W15}{2}
    \gantttitle{W16}{2}
    \gantttitle{W17}{2}
    \gantttitle{W18}{2}
    \gantttitle{W19}{2}
    \gantttitle{W20}{2}
    \gantttitle{W21}{2}
    \gantttitle{W22}{2}
    \gantttitle{W23}{2}
    \gantttitle{W24}{2}
    \gantttitle{W25}{2}
    % Setting group if any
    \ganttgroup[inline=false]{Documents}{1}{16} \\
    \ganttbar[progress=30,inline=false]{Requirement}{1}{5} \\
    \ganttbar[progress=0,inline=false]{Technical}{5}{10}\\
    \ganttbar[progress=0,inline=false]{Design}{10}{15} \\

    \ganttgroup[inline=false]{\small  {Winter Break} } {17}{24} \\

    \ganttgroup[inline=false]{\small{Desing Game Interface}}{25}{30} \\

    \ganttgroup[inline=false]{\small{Train Neural Network}}{31}{39} \\
    \ganttbar[progress=0,inline=false]{Training 1}{31}{33}\\
    \ganttbar[progress=0,inline=false]{Training 2}{34}{36}\\
    \ganttbar[progress=0,inline=false]{Training 3}{37}{39}\\

    \ganttgroup[inline=false]{Game Testing}{31}{39} \\
    \ganttbar[progress=0,inline=false]{Test 1}{32}{33}\\
    \ganttbar[progress=0,inline=false]{Test 2}{35}{36}\\
    \ganttbar[progress=0,inline=false]{Test 3}{38}{41}\\

    \ganttgroup[inline=false]{Improvements}{42}{50} \\

\end{ganttchart}
%    \caption{Gantt diagram for 2013--2014 Project}
%\end{figure}
}
% \end{turn}{90}

\end{document}

\grid
