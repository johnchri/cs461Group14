\documentclass[letterpaper,10pt]{article}
\usepackage{graphicx}
\usepackage{amssymb}
\usepackage{amsmath}
\usepackage{amsthm}

\usepackage{alltt}
\usepackage{float}
\usepackage{color}
\usepackage{url}

\usepackage{enumitem}

\usepackage{geometry}

\usepackage{titling}
\usepackage{rotating}
\usepackage{pgfgantt}
\usepackage{graphicx}
\usepackage{xcolor}
\usepackage{anyfontsize}

\ganttset{group/.append style={orange},
milestone/.append style={red},
progress label node anchor/.append style={text=red}}

\geometry{textheight=8.5in, textwidth=6in}

%\usepackage{hyperref}

\def\name{Christopher Johnson, Gabe Morey, Luay Alshawi}

\parindent = 0.0 in
\parskip = 0.1 in

\pretitle{\begin{center}\Huge\bfseries}
\posttitle{\par\end{center}\vskip 0.5em}
\preauthor{\begin{center}\Large\ttfamily}
\postauthor{\end{center}}
\author{Christopher Johnson, Gabe Morey, Luay Alshawi}
\title{Requirements Documents}

\date{October 27, 2016}

\begin{document}

\begin{titlingpage}
\maketitle
CS461 Fall Term

\begin{abstract}

\end{abstract}

\end{titlingpage}
Our team name will be AI Gaming.


The project must use a deep learning neural net to accomplish a task.
The machine will take video input and use it to learn how to play the game Galaga.
A camera or screen capture device must be used to retrieve game state input.
The neural net must be trained and allowed to create its own code to accomplish the task.
The neural network should define for itself how to respond to in-game stimuli and react accordingly.
We may code interactions between the net and the game to allow it to play the game.
We may not hard code its responses to in-game stimuli, as these responses must be trained into the neural net.
The final project must be able to run inference on the Jetson TX1 developer platform.

The solution of our project will need to be able to beat the first level of the game at least 90\% of the time.
Ideally the system should be able to make it past level five approximately 50\% of the time.
The ability of the system to make it farther than that on average, or to increase the percent chance on it making it to certain points, will be considered a stretch goal.

We must clearly outline our actions taken and resources required at every step of the process.
This means that all technology used to accomplish our task must be recorded and accounted for.
Our process will require a 

% \begin{turn}{90}
\resizebox{\textwidth}{!}{

%    \centering
     \begin{ganttchart}[%Specs
     y unit title=2cm,
     y unit chart=2cm,
     vgrid,hgrid,
     title height=1,
%     title/.style={fill=none},
     title label font=\bfseries\footnotesize,
     bar/.style={fill=blue},
     bar height=0.7,
%   progress label text={},
     group right shift=0,
     group top shift=0.7,
     group height=.3,
     group peaks width={0.2},
     inline]{1}{50}
    %labels
    \gantttitle{Plan}{50}\\  % title 1
    \gantttitle{W1}{2}
    \gantttitle{W2}{2}
    \gantttitle{W3}{2}
    \gantttitle{W4}{2}
    \gantttitle{W5}{2}
    \gantttitle{W6}{2}
    \gantttitle{W7}{2}
    \gantttitle{W8}{2}
    \gantttitle{W9}{2}
    \gantttitle{W10}{2}
    \gantttitle{W11}{2}
    \gantttitle{W12}{2}
    \gantttitle{W13}{2}
    \gantttitle{W14}{2}
    \gantttitle{W15}{2}
    \gantttitle{W16}{2}
    \gantttitle{W17}{2}
    \gantttitle{W18}{2}
    \gantttitle{W19}{2}
    \gantttitle{W20}{2}
    \gantttitle{W21}{2}
    \gantttitle{W22}{2}
    \gantttitle{W23}{2}
    \gantttitle{W24}{2}
    \gantttitle{W25}{2}
    % Setting group if any
    \ganttgroup[inline=false]{Documents}{1}{16} \\
    \ganttbar[progress=30,inline=false]{Requirement}{1}{5} \\
    \ganttbar[progress=0,inline=false]{Technical}{5}{10}\\
    \ganttbar[progress=0,inline=false]{Design}{10}{15} \\

    \ganttgroup[inline=false]{\small  {Winter Break} } {17}{24} \\

    \ganttgroup[inline=false]{\small{Desing Game Interface}}{25}{30} \\

    \ganttgroup[inline=false]{\small{Train Neural Network}}{31}{39} \\
    \ganttbar[progress=0,inline=false]{Training 1}{31}{33}\\
    \ganttbar[progress=0,inline=false]{Training 2}{34}{36}\\
    \ganttbar[progress=0,inline=false]{Training 3}{37}{39}\\

    \ganttgroup[inline=false]{Game Testing}{31}{39} \\
    \ganttbar[progress=0,inline=false]{Test 1}{32}{33}\\
    \ganttbar[progress=0,inline=false]{Test 2}{35}{36}\\
    \ganttbar[progress=0,inline=false]{Test 3}{38}{41}\\

    \ganttgroup[inline=false]{Improvements}{42}{50} \\

\end{ganttchart}
%    \caption{Gantt diagram for 2013--2014 Project}
%\end{figure}
}
% \end{turn}{90}

\end{document}

\grid
