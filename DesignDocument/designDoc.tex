%Copyright 2014 Jean-Philippe Eisenbarth
%This program is free software: you can
%redistribute it and/or modify it under the terms of the GNU General Public
%License as published by the Free Software Foundation, either version 3 of the
%License, or {at your option} any later version.
%This program is distributed in the hope that it will be useful,but WITHOUT ANY
%WARRANTY; without even the implied warranty of MERCHANTABILITY or FITNESS FOR A
%PARTICULAR PURPOSE. See the GNU General Public License for more details.
%You should have received a copy of the GNU General Public License along with
%this program.  If not, see <http://www.gnu.org/licenses/>.

%Based on the code of Yiannis Lazarides
%http://tex.stackexchange.com/questions/42602/software-requirements-specification-with-latex
%http://tex.stackexchange.com/users/963/yiannis-lazarides
%Also based on the template of Karl E. Wiegers
%http://www.se.rit.edu/~emad/teaching/slides/srs_template_sep14.pdf
%http://karlwiegers.com
% \documentclass[letterpaper,10pt]{article}
\documentclass{scrreprt}
\usepackage{listings}
\usepackage{underscore}
\usepackage[bookmarks=true]{hyperref}
\usepackage[utf8]{inputenc}
\usepackage[english]{babel}
\usepackage{graphicx}
\usepackage{amssymb}
\usepackage{amsmath}
\usepackage{amsthm}
\usepackage{afterpage}

\usepackage{alltt}
\usepackage{float}
\usepackage{color}
\usepackage{url}

\usepackage{enumitem}

\usepackage{geometry}

\usepackage{titling}
\usepackage{rotating}
\usepackage{pgfgantt}
\usepackage{graphicx}
\usepackage{xcolor}
\usepackage{anyfontsize}

\ganttset{group/.append style={orange},
milestone/.append style={red},
progress label node anchor/.append style={text=red}}

\newcommand\blankpage{%
    \null
    \thispagestyle{empty}%
    \addtocounter{page}{-1}%
    \newpage}

\geometry{textheight=8.5in, textwidth=6in}
\hypersetup{
    bookmarks=false,    % show bookmarks bar?
    pdftitle={Software Requirement Specification},    % title
    pdfauthor={Jean-Philippe Eisenbarth},                     % author
    pdfsubject={TeX and LaTeX},                        % subject of the document
    pdfkeywords={TeX, LaTeX, graphics, images}, % list of keywords
    colorlinks=true,       % false: boxed links; true: colored links
    linkcolor=blue,       % color of internal links
    citecolor=black,       % color of links to bibliography
    filecolor=black,        % color of file links
    urlcolor=purple,        % color of external links
    linktoc=page            % only page is linked
}%
\def\myversion{1.0 }
\date{}
%\title
\usepackage{hyperref}
\begin{document}

\begin{flushright}
    \rule{16cm}{5pt}\vskip1cm
    \begin{bfseries}
        \Huge{Design Document}\\
        \vspace{1.0cm}
        Deep Learning on Embedded Platform\\
        \vspace{1.0cm}
        \LARGE{Version \myversion}\\
        \vspace{1.0cm}
        Prepared by Christopher Johnson, Luay Alshawi, Gabe Morey\\
        \vspace{1.0cm}
        CS 461 Fall 2016\\
        \vspace{1.0cm}
        \today\\
    \end{bfseries}
\end{flushright}

This document goes over the design decision made for our Deep Learning on Embedded Platform project.
In it we will explain our approach, provide detailed information about why we designed it a certain way, and explain our expected outcome.

\afterpage{\blankpage}

\tableofcontents

\chapter{Overview}
\section{Scope}

\section{Purpose}

The purpose of this document is to lay out the ground work for the design of our project plan out the process.

\section{Intended Audience}

The intended audience of this document is the client who's project we are designing, as well as the instructors and TAs guiding us through the process.

\section{Conformance}



\chapter{Definitions}

Deep Learning or Machine Learning - Using
API - Application Programming Interface
Jetson TX1 - Quad core embedded system designed for power efficiency and deep learning projects
OS - Operating system

\chapter{Design}

\section{Introduction}

\section{Gabe}

\section{Luay}

\section{Working with the Jetson}

While working with the Jetson TX1 there are several things that we needed to decide.
These included our methods to communicate with our client, and the hardware used with the Jetson.
For communication purposes, our team decided to stick with using skype.
We will conduct regular meetings with our client and continue using skype, which everyone is familiar with.
Skype will allow us to video call as well, in case any visuals need to be explained.

Our project will actually need to be able to control the game.
Inputs from the neural net need to be received by the game so that it can be played.
To do this we decided not to use the standard keyboard.
This would require a lot of additional coding to map the Jetson to the controls.
Instead we decided to find a game controller that could be hardwired to the Jetson TX1.
A game controller would significantly reduce the work load because it would already contain a lot of built in functionality that we would need to connect it to the Jetson.
The Jetson will be able to control the game using the controller's built in functionality.

Since the Jetson has to learn how to play the game it needs to be able to see the game.
For this we decided to use the camera sent to us by our client.
This seemed to be the simplest option seeing as we could connect the camera directly to the Jetson.
The Jetson

\chapter{Conclusion}



\end{document}