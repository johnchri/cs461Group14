%Copyright 2014 Jean-Philippe Eisenbarth
%This program is free software: you can
%redistribute it and/or modify it under the terms of the GNU General Public
%License as published by the Free Software Foundation, either version 3 of the
%License, or {at your option} any later version.
%This program is distributed in the hope that it will be useful,but WITHOUT ANY
%WARRANTY; without even the implied warranty of MERCHANTABILITY or FITNESS FOR A
%PARTICULAR PURPOSE. See the GNU General Public License for more details.
%You should have received a copy of the GNU General Public License along with
%this program.  If not, see <http://www.gnu.org/licenses/>.

%Based on the code of Yiannis Lazarides
%http://tex.stackexchange.com/questions/42602/software-requirements-specification-with-latex
%http://tex.stackexchange.com/users/963/yiannis-lazarides
%Also based on the template of Karl E. Wiegers
%http://www.se.rit.edu/~emad/teaching/slides/srs_template_sep14.pdf
%http://karlwiegers.com
% \documentclass[letterpaper,10pt]{article}
\documentclass{scrreprt}
\usepackage{listings}
\usepackage{underscore}
\usepackage[bookmarks=true]{hyperref}
\usepackage[utf8]{inputenc}
\usepackage[english]{babel}
\usepackage{graphicx}
\usepackage{amssymb}
\usepackage{amsmath}
\usepackage{amsthm}
\usepackage{afterpage}

\usepackage{alltt}
\usepackage{float}
\usepackage{color}
\usepackage{url}

\usepackage{enumitem}

\usepackage{geometry}

\usepackage{titling}
\usepackage{rotating}
\usepackage{pgfgantt}
\usepackage{graphicx}
\usepackage{xcolor}
\usepackage{anyfontsize}

\ganttset{group/.append style={orange},
milestone/.append style={red},
progress label node anchor/.append style={text=red}}

\newcommand\blankpage{%
    \null
    \thispagestyle{empty}%
    \addtocounter{page}{-1}%
    \newpage}

\geometry{textheight=8.5in, textwidth=6in}
\hypersetup{
    bookmarks=false,    % show bookmarks bar?
    pdftitle={Software Requirement Specification},    % title
    pdfauthor={Jean-Philippe Eisenbarth},                     % author
    pdfsubject={TeX and LaTeX},                        % subject of the document
    pdfkeywords={TeX, LaTeX, graphics, images}, % list of keywords
    colorlinks=true,       % false: boxed links; true: colored links
    linkcolor=blue,       % color of internal links
    citecolor=black,       % color of links to bibliography
    filecolor=black,        % color of file links
    urlcolor=purple,        % color of external links
    linktoc=page            % only page is linked
}%
\def\myversion{1.0 }
\date{}
%\title
\usepackage{hyperref}
\begin{document}

\begin{flushright}
    \rule{16cm}{5pt}\vskip1cm
    \begin{bfseries}
        \Huge{Design Document}\\
        \vspace{1.9cm}
        for\\
        \vspace{1.9cm}
        Deep Learning on Embedded Platform\\
        \vspace{1.9cm}
        \LARGE{Version \myversion}\\
        \vspace{1.9cm}
        Prepared by Christopher Johnson, Luay Alshawi, Gabe Morey\\
        \vspace{1.9cm}
        CS 461 Fall 2016\\
        \vspace{1.9cm}
        \today\\
    \end{bfseries}
\end{flushright}
\afterpage{\blankpage}

\tableofcontents

\chapter{Overview}
\section{Scope}

\section{Purpose}

\section{Intended Audience}

\section{Conformance}

\chapter{Definitions}

Deep Learning or Machine Learning - Using
API - Application Programming Interface

\chapter{Conceptual Model for Software Design Descritions}

\section{Software Design in Context}

\section{Siftware Design Description within the life cycle}

\subsection{Influences on SDD preparation}

\subsection{Influences on software life cycle products}

\subsection{Design verification and design role in validation}

\chapter{Design descrition information content}

\section{Introducation}

\section{SSD identification}

\section{Design stakeholders and their concerns}

\section{Design views}

\section{Design viewpoints}

\section{Design elements}

\subsection{Design entities}

\subsection{Design attributes}

\subsection{Name attribute}

\subsection{Type attribute}

\subsection{Pupose attribute}

\subsection{Author Attribute}

\subsection{Design relationships}

\subsection{Design constraints}

\section{Design overlays}

\section{Design rationale}

\section{Design language}

\chapter{Design viewpoints}

\section{Introducation}

\section{Context viewpoint}

\section{Composition viewpoint}

\section{Logical viewpoint}

\section{information viewpoint}

\section{Pattns use viewpoint}

\section{Interface viewpoint}

\section{Structure viewpoint}

\section{Interaction viewpoint}

\section{State dynamics viewpoint}

\section{Algorithm viewpoint}

\section{Resource viewpoint}




\end{document}